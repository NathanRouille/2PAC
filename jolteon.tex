\documentclass{article}
\usepackage[utf8]{inputenc}
\usepackage{amsmath}
\usepackage{stmaryrd}
\usepackage{geometry}

\geometry{top=1cm,bottom=2cm}

\title{Preuve de la Safety du Consensus Jolteon}
\author{}
\date{}

\begin{document}

\maketitle
\vspace{-3\baselineskip}

\section*{Notation}
\begin{itemize}
    \item \(B_i \leftarrow QC_i \leftarrow B_{i+1}\) signifie que \(B_i\) est certifié par \(QC_i\) qui est compris dans \(B_{i+1}\) (\(B_{i+1}\) est le block précédent \(B_i\) mais pas nécessairement à 2 rounds consécutifs)
    \item \(B_i \leftarrow^* B_j\) signifie que \(B_j\) étend \(B_i\) i.e. il existe une chaîne :\\ \(B_i \leftarrow QC_i \leftarrow B_{i+1} \leftarrow QC_{i+1} \leftarrow [\ldots] \leftarrow B_{j-1} \leftarrow QC_{j-1} \leftarrow B_j\) \\(un block s'étend lui-même : pour tout block \(B\), on a \(B \leftarrow^* B\))
\end{itemize}

\section*{Définition Global Direct-Commit}
Un block \(B\) est dit globally direct-committed (GDC) si \(f+1\) replicas \underline{honnêtes} votent pour une proposition de block \(B'\) au round \(B.r + 1\) tel que \(B'.qc\) certifie \(B\)

\section*{Théorème}
Soit \(B\) un block GDC, pour tout block certifié \(B'\) tel que \(B'.r \geq B.r\), on a : \(B \leftarrow^* B'\)
\\
\\
\textbf{Preuve:}\\
Soit \(B\) un GDC, on note \(r_0 := B.r\)\\
Pour tout \(r \geq r_0\), on note :\\ \(P(r)\): ``Pour tout block \(B'\) certifié au round \(r\), on a \(B \leftarrow^* B'\) ''\\
On procède par récurrence forte:
\begin{itemize}
        \item Initialisation : pour \(r = r_0\), comme \(B\) est GDC, il est par définition certifié. Il est donc le seul block certifié au round \(r_0\). Or \(B \leftarrow^* B\) par définition du formalisme. Ainsi \(P(r_0)\) est vraie.
        \item Hérédité : Soit \(r' > r_0\), on suppose que \(P(r)\) est vraie pour tout\\ \(r \in \llbracket r_0, r'-1 \rrbracket\), montrons que \(P(r')\) est vraie :
            \begin{itemize}
                \item Si aucun block n'est certifié au round \(r'\), alors \(P(r')\) est vraie.
                \item Si un block \(B'\) est certifié au round \(r\), alors il existe un block \(B''\) certifié au round \(r''\in\llbracket r_0, r'-1 \rrbracket\) tel que \(B'' \leftarrow QC_{B''} \leftarrow B'\) donc par hypothèse de récurrence, \(P(r'')\) est vraie, et on a \\ \(B \leftarrow^* B'' \leftarrow QC_{B''} \leftarrow B'\) d'où \(B \leftarrow^* B'\) et donc \(P(r')\) est vraie :
                \begin{itemize}
                    \item En effet, \(B'\) étant certifié ne peut étendre qu'un block certifié d'un round strictement inférieur, donc \(r'' < r'\)
                    \item Par ailleurs, si \(r''<r_0\), alors \(r' > r''+1\) donc \(B'\) contient un TC (nouveau formalisme contenant les $r_{\text{lock}}$ signés) avec \(2f+1\) $r_{\text{lock}}$ de replicas. Comme \(B\) est GDC, au moins un de ces $r_{\text{lock}}$ appartient à un des \(f+1\) replicas honnêtes ayant voté pour le block étendant \(B\) au round \(r_0+1\) et donc ce $r_{\text{lock}}$ est supérieur ou égal à \(r_0\). Cette actualisation du $r_{\text{lock}}$ est forcément antérieure au TC qui a été formé au round \(r'-1\)\(\geq r_0\) car il n'a pas pu être formé au round \(r_0\) où \(B\) a été certifié et s'il a été formé au round \(r_0 +1\) c'est nécessairement après que les f+1 replicas honnêtes aient voté pour le block étendant \(B\) car ils ne peuvent pas voter après avoir émis leur timeout message. Ainsi les \(2f+1\) replicas honnêtes ne pourraient pas voter pour \(B'\) car\\ \(r'' = B'.qc.r < r_0 \leq \max(r_{\text{lock}} \in B'.tc)\) et donc \(B'\) ne pourrait pas être certifié. Or \(B'\) est certifié donc par l'absurde : \(r'' \geq r_0\).
                \end{itemize}
                
            \end{itemize}
        \item Conclusion : La propriété \(P\) est initialisée à \(r_0\) et est héréditaire donc par récurrence forte elle est vraie pour tout round \(r \geq r_0\).
\end{itemize}

\section*{Corollaire}
Soient \(B\) et \(B'\) tous deux GDC, alors on a :
\[
\left\{
\begin{array}{l}
B \leftarrow^* B' \quad \\ \text{ou} \\
B' \leftarrow^* B
\end{array}
\right.
\]
\\
\\
\textbf{Preuve:}\\
Par définition, GDC \(\Rightarrow\) certifié et on a toujours : \(B.r \geq B'.r\) ou \(B'.r \geq B.r\)

\section*{Safety}
Un block certifié (et tous ses ancêtres) est commit s'il est étendu par un block certifié consécutif. Donc par définition, commit \(\Rightarrow\) global direct-commit car parmi les \(2f+1\) votants au plus \(f\) sont byzantins donc au moins \(f+1\) sont honnêtes. Ainsi, grâce au corollaire précédent, à chaque fois qu'un replica commit, il a la certitude que le dernier block qu'il commit est GDC et donc qu'il existe une chaîne, unique car chaque block n'a qu'un parent, entre ce block et le dernier qu'il avait commit, lui aussi GDC et donc par récurrence une unique chaîne jusqu'au Genesis block qui est par définition commit. Cette chaîne, qui une fois commit est inaltérable, est commune au moins aux \(2f+1\) replicas honnêtes qui ont exécuté de manière identique le consensus. On obtient alors une blockchain sécurisée, ce qui prouve donc la safety du Consensus Jolteon.

\end{document}
